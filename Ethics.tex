\section{Ethics}
There has been much discussion about 









\section{Truth Values}
First, all statements \textit{s}, have a truth value \textit{t}, where 0 means that a statement is universally false and 1 is a statement that is universally true, such that:
\begin{equation}
\forall s \in S,\exists t : 0 \leq t \leq 1 
\end{equation}
To determine these truth values, we should separate all statements into four groups: \textit{analytic a priori}, \textit{analytic a posteriori}, \textit{synthetic a priori}, and \textit{synthetic a posteriori} \citep{wright1997companion}.
\subsection{Analytic a Priori}
An \textit{analytic a priori} statement's truth value is "true by virtue of meanings and independent of fact" \citep{quine1951main}. 
For example: $S_1$: All triangles have three sides.
The predicate "having three sides" is contained within the meaning of the word "triangle". There is no need to observe a given triangle in order to know that it must have three sides.

For the purposes of this discussion, racial slurs and other such language shall be considered \textit{analytic a priori} hate speech, as there is no need to observe all racial slurs in order to know that they were created and used with hate speech as the intent.

\subsection{Analytic a Posteriori}
A statement is \textit{analytic a posteriori} if it is necessarily true (always true and not contingent upon truth values separate to the proposition at hand), but requires empirical evidence to discover it \citep{kripke1972naming}. Kripke gives the example of Venus being both the morning and evening star: 
$S_2$: Venus is the morning star.
$S_3$: Venus is the evening star.
$S_4$: The morning star is the evening star.

While this statement is necessarily (always) true, it can't be considered \textit{a priori} true, as Homer (12th century BCE) refers to the morning and evening stars as separate objects, and it isn't until Pythagoras (500 BCE) that it is a scientifically accepted fact that the morning and evening stars refer to the same object, Venus \citep{dunne1978voyage}.

Scientific misinformation would largely fall into this category. Statements such as 
$S_5$: The Earth is round 
are necessarily true, but requires scientific evidence to understand that roundness is a requirement of being the planet Earth.  

Determining the truth values of \textit{analytic} statements are perfect opportunities for classification regression, and much of the literature on detecting fake news revolves around detecting analytic statements that are false: spam detection is \textit{analytic a posteriori}, for example. There is no need to read every Nigerian prince scam to know that they are all untrue, but a general knowledge of the world and observation are required to know that this set is untrue. 