%% Documentclass:
%% Network Neuroscience
\documentclass[NETN,manuscript]{stjour-new}

%%%%%%%%%%% Article Set-Up %%%%%%%%%%%%%%%%%%%%%%%%%%%%%%%%%%%%
%% Article Type:
%% Default is Research.

%% Or, choose one of these options:
%% Research, Methods, Data, Review, and Perspective

\articletype{Perspective}

%%%%%%%%%%%%%%%%%%%%%%%%%%%%%%%%%%%%%%%%%%%%%%%%%%%%%%%%%%%%%%%
%% author definitions should be placed here:

%% example definition
\def\taupav{\tau_{\mathrm{Pav}}}

%% common path to your figure files (for organization purposes)
\graphicspath{{figs/}}

\begin{document}
\title[Shortened Version of Title]{Title of Article}
\subtitle{Subtitle Here}

%% If shortened title for running head is needed so that the article title can fit
%%   in the running head, use [] argument, ie,
%%
%%   \title[Shortened Title for Running Head]{Title of Article}
%%   \subtitle{Subtitle Here}

%% Since we use \affil{} in the argument of the \author command, we
%% need to supply another version of the author names, without \affil{}
%% to be used for running heads:

\author[John Hawthorne Smith]
{John Hawthorne Smith\affil{1}}

\affiliation{1}{Masters of Science in Data Science, Northwestern University, Evanston, United States}

%ie.
%\affiliation{1}{Gatsby Computational Neuroscience Unit, University
%College London, London, United Kingdom} 


\correspondingauthor{John Hawthorne Smith}{JohnSmith2020@u.northwestern.edu}

% ie,
%\correspondingauthor{Ritwik K. Niyogi}{ritwik.niyogi@gatsby.ucl.ac.uk}

\keywords{(a series of capitalized words, separated with commas)}

%ie
%\keywords{Work, Leisure, Normative, Microscopic,  Reinforcement Learning, Economics}

\begin{abstract}
Abstract text here.
\end{abstract}

\begin{authorsummary}
Author summary here. 
\end{authorsummary}


\section{Sample Section}
Text here. Text here. Text here. Text here.
Text here. Text here. Text here. Text here.
Text here. Text here. Text here. Text here.
Text here. Text here. Text here. Text here.

\subsection{Sample Subsection}
Text here. Text here. Text here. Text here.
Text here. Text here. Text here. Text here.
Text here. Text here. Text here. Text here.
Text here. Text here. Text here. Text here.

\subsubsection{Sample Subsubsection}
Text here. Text here. Text here. Text here.
Text here. Text here. Text here. Text here.
Text here. Text here. Text here. Text here.
Text here. Text here. Text here. Text here.


\section{Sample equations}
\begin{equation}
\label{eq:rhoCHT}
\rho^{\pi}= \frac{RI + \mathbb{E}_{\pi([L,\tau_L]|\textrm{post})}
\left[C_L(\taupav+\tau_L) \right]   +
\displaystyle{\int_{0}^{P}}{dw~ \mathbb{E}_{\pi_{w_L}}}
\Biggl[\/\sum_{n_{L|[\textrm{pre},w]}}C_L(\tau_L)
\Biggr]            }      {P +
\mathbb{E}_{\pi([L,\tau_L] |\textrm{post})}[\tau_{L}] +\taupav +
\displaystyle{ \int_{0}^{P}}{dw~ \mathbb{E}_{\pi_{w_L}}}   
\Biggl[\sum_{n_{L|[\textrm{pre},w]}}\tau_L\Biggr]  
}
\end{equation}
As long as
$RI - K_LP > 
\frac{1}{\beta}$
\begin{equation}
%\def\theequation{5.1}
\left.\begin{array}{lrcl}
&\rho^{\pi} &=&  \displaystyle\frac{\beta ( RI + K_L \taupav )-1} {\beta
(P+\taupav )}    \\[12pt]
\hbox{and}\hbox to .25in{\hfill}&\mathbb{E}[\tau_L | \text{post}] &=&\displaystyle \frac{P+\taupav}{\beta ( RI -
K_LP)-1}  
\label{eq:analytical_linear}
\end{array}\right\}\hbox to 1.25in{\hfill}
\end{equation} 
Text finishing first page.
Text finishing first page.
Text finishing first page.
Text finishing first page.
Text finishing first page.
Text finishing first page.
Text finishing first page.


\begin{boxedtext}{Comparative Analysis of Different Classes of Networks} 
Going beyond the examination of shared topological features across
nervous systems, the generalized mathematical language of graph theory
also offers tools for the comparison of the organization of brain
networks to other classes of network studied
by different scientific
disciplines. 

Many real-world systems operate as some sort of
interaction or communication network, including, for example, social
networks, gene regulatory networks, computer networks, and
transportation networks. Similar to brain networks, many of these
real-world networks display an efficient small-world organization, a
pronounced community structure with densely connected modules, as well
as the formation of hubs and rich clubs. Going beyond the
comparison of networks within the class of nervous systems, the field
of `comparative network analysis' examines commonalities and
differences across a range of network classes.

\begin{figure}
\includegraphics[width=\hsize]{Fig1}
\caption{Here is the caption.}
\end{figure}
\end{boxedtext}

\newpage
\begin{boxedtext}{Comparative Analysis of Different Classes of Networks} 
Going beyond the examination of shared topological features across
nervous systems, the generalized mathematical language of graph theory
also offers tools for the comparison of the organization of brain
networks to other classes of network studied by different scientific
disciplines. Many real-world systems operate as some sort of
interaction or communication network, including, for example, social
networks, gene regulatory networks, computer networks, and
transportation networks. Similar to brain networks, many of these
real-world networks display an efficient small-world organization, a
pronounced community structure with densely connected modules, as well
as the formation of hubs and rich clubs. Going beyond the
comparison of networks within the class of nervous systems, the field
of `comparative network analysis' examines commonalities and
differences across a range of network classes.

\begin{table}[ht]
\caption{Here is the caption.}

\centerline{\begin{tabular}{|c|c|c|}
\hline
one&two&three\\
\hline
four&five&six\\
\hline
\end{tabular}}
\end{table}
\end{boxedtext}

\subsection{Technical Terms}
All NETN article types require Technical Terms.

Identify approximately 10 key terms that are mentioned in your article and whose usage and definition may not be familiar across the broad readership of the journal. 
Provide brief (20-word or less) definitions for each term, avoiding in these definitions the use of jargon, or highly technical or specialized language. 
When the article is typeset, the Technical Terms will appear in the margins at or near their first mention in the text.

In your manuscript, bold the first occurrence of each \textbf{Technical Term} and then provide a list of the terms and their definitions at the end of the manuscript after the references. 

\subsection{Simple code sample}

\begin{code}
\begin{verbatim}
procedure bubbleSort( A : list of sortable items )
    n = length(A)
    repeat
       newn = 0
       for i = 1 to n-1 inclusive do
          if A[i-1] > A[i] then
             swap(A[i-1], A[i])
             newn = i
          end if
       end for
       n = newn
    until n = 0
end procedure
\end{verbatim}
\end{code}


\subsection{Algorithm environment}
%% \begin{algorithm} takes option [p][b][t][h],  or some combination, like \begin{figure}
%% See documentation for algorithmic.sty for more information on formatting algorithms.

\begin{algorithm}[h]
\caption{A sample in an algorithm environment.}
\begin{algorithmic}
\If {$i\geq maxval$}
    \State $i\gets 0$
\Else
    \If {$i+k\leq maxval$}
        \State $i\gets i+k$
    \EndIf
\EndIf
\end{algorithmic}
\end{algorithm}


\section{Itemized Lists}

\subsection{Roman list:}

\begin{enumerate}
\item[(i)] at high 
payoffs, subjects work almost continuously.
\item[(ii)] at low payoffs, they 
engage in leisure all at once, in long bouts after working.
\item[(iii)] subjects work continuously for the entire price duration, as long as
the price is not very long;
\item[(iv)] the duration of leisure bouts is variable.
\end{enumerate}


\subsection{Numbered list:}

\begin{enumerate}
\item at high 
payoffs, subjects work almost continuously, engaging in little leisure
inbetween work bouts; 
\item at low payoffs, they 
engage in leisure all at once, in long bouts after working, rather
than distributing the same amount of leisure time into multiple short
leisure bouts; 
\item subjects work continuously for the entire price duration, as long as
the price is not very long (as shown by an analysis conducted by
Y-AB, to be published separately);  
\item the duration of leisure bouts is variable.
\end{enumerate}


\subsection{Bulleted list:}

\begin{itemize}
\item at high 
payoffs, subjects work almost continuously, engaging in little leisure
inbetween work bouts; 
\item at low payoffs, they 
engage in leisure all at once, in long bouts after working, rather
than distributing the same amount of leisure time into multiple short
leisure bouts; 
\item subjects work continuously for the entire price duration, as long as
the price is not very long (as shown by an analysis conducted by
Y-AB, to be published separately);  
\item the duration of leisure bouts is variable.
\end{itemize}

\subsection{Description list:}
\begin{description}
\item[High payoffs:] at high 
payoffs, subjects work almost continuously, engaging in little leisure
inbetween work bouts; 
\item[Low payoffs:] at low payoffs, they 
engage in leisure all at once, in long bouts after working, rather
than distributing the same amount of leisure time into multiple short
leisure bouts; 
\item[Continuous work:] subjects work continuously for the entire price duration, as long as
the price is not very long (as shown by an analysis conducted by Y-AB, to be published separately); 
\item[Duration:] the duration of leisure bouts is variable.
\end{description}

\newpage
\section{Sample citations}
For general information on the correct form for citations using
the APA 6 format, see the following sites:
\href{https://owl.english.purdue.edu/owl/resource/560/02/}
{APA 6, In-text citations, The Basics} and
\href{https://owl.english.purdue.edu/owl/resource/560/03/}
{APA 6, In-text citations}



\section{Natbib citation mark up}

\subsection{Single citations}
\noindent
\begin{tabular}{ll}
\bf Type&\bf Results\\
\hline
\verb+\citet{jon90}+&\dogray Jones et al. (1990)\\
\verb+\citet[chap. 2]{jon90}+&\dogray Jones et al. (1990, chap. 2)\\
    \verb+\citep{jon90}+	    &\dogray   	(Jones et al., 1990)\\
    \verb+\citep[chap. 2]{jon90}+ 	&\dogray    	(Jones et al., 1990, chap. 2)\\
    \verb+\citep[see][]{jon90}+ 	 &\dogray    	(see Jones et al., 1990)\\
    \verb+\citep[see][chap. 2]{jon90}+ 	&\dogray    	(see Jones et al., 1990, chap. 2)\\
    \verb+\citet*{jon90}+ 	    &\dogray    	Jones, Baker, and Williams (1990)\\
    \verb+\citep*{jon90}+	    & \dogray   	(Jones, Baker, and Williams,
    1990) \\
\end{tabular}

For example, some citations from the NETNbibsamp.bib database:




citet: \citet{bullmore2009complex}, citep: \citep{gomez2009analysis}, and
citep*: \citep*{de2012cortical}

\subsection{Multiple citations}
Multiple citations may be made by including more than one citation
key in the \verb+\cite+ command argument.

\noindent
\begin{tabular}{ll}
\bf Type&\bf Results\\
\hline
\verb+\citet{jon90,jam91}+&\dogray Jones et al. (1990); James et al. (1991)\\
\verb+\citep{jon90,jam91}+&\dogray (Jones et al., 1990; James et al. 1991)\\
\verb+\citep{jon90,jon91}+&\dogray (Jones et al., 1990, 1991)\\
\verb+\citep{jon90a,jon90b}+&\dogray (Jones et al., 1990a,b)\\
\end{tabular}

For example, multiple citations from the bibsamp.bib database:
citet: \citet{nooner2012nki,hutchison2013resting},
 citep:
\citep{tagliazucchi2012dynamic, de2012cortical}

As you see, the citations are automatically hyperlinked to their
reference in the bibliography.

\newpage

\section{Sample figures}

\begin{figure}[h] 
\centerline{\includegraphics[width=\textwidth]{Fig1.pdf}}
\caption{(Colour online) \textbf{Task and key features of the
 data.} \\
 A) Cumulative handling time (CHT) task. Grey bars denote work
(depressing a lever), white gaps show leisure. The subject must
accumulate work up to a total period of time called the
\emph{price} ($P$) in order to obtain a single reward (black dot) of subjective reward
intensity $RI$. The trial duration is $25\times \mathrm{price}$ (plus
$2$s each time the price is attained, during which the lever is retracted so it cannot
work; not shown).}
\label{fig:task_data}
\end{figure}
%% this command ends a page but does not fill the bottom with white space:
\eject

\begin{figure}[ht] 
\widefigure{\fullpagewidth}{Fig1.pdf}
\caption{(Colour online) \textbf{Task and key features of the
 data.} \\
 A) Cumulative handling time (CHT) task. Grey bars denote work
(depressing a lever), white gaps show leisure. The subject must
accumulate work up to a total period of time called the
\emph{price} ($P$) in order to obtain a single reward (black dot) of subjective reward
intensity $RI$. The trial duration is $25\times \mathrm{price}$ (plus
$2$s each time the price is attained, during which the lever is retracted so it cannot
work; not shown).}
\label{fig:task_data2}
\end{figure}

\newpage
\section{Sample tables}

\begin{table}[!ht]
\caption{Time of the Transition Between Phase 1 and Phase 2$^{a}$}
\label{tab:label}
\centering
\begin{tabular}{lc}
\hline
 Run  & Time (min)  \\
\hline
  $l1$  & 260   \\
  $l2$  & 300   \\
  $l3$  & 340   \\
  $h1$  & 270   \\
  $h2$  & 250   \\
  $h3$  & 380   \\
  $r1$  & 370   \\
  $r2$  & 390   \\
\hline
\multicolumn{2}{l}{$^{a}$Table note text here.}
\end{tabular}
\end{table}

\begin{table}[ht]
\widecaption{Sample table taken from [treu03]\label{tbl-1}}
\begin{widetable}
\advance\tabcolsep-1pt
\footnotesize
\begin{tabular}{ccrrccccccccc}
\hline
\bf 
POS &\bf  chip &\multicolumn1c{\bf ID} &\multicolumn1c{\bf X}
&\multicolumn1c{\bf Y} &\bf
RA &\bf DEC &\bf IAU$\pm$ $\delta$ IAU &\bf
IAP1$\pm$ $\delta$ IAP1 &\bf IAP2 $\pm$ $\delta$
IAP2 &\bf star &\bf E &\bf Comment\\
\hline
0 & 2 & 1 & 1370.99 & 57.35\rlap{$^a$}    &   6.651120 &  17.131149 &
21.344$\pm$0.006\rlap{$^b$}  & 2 4.385$\pm$0.016 & 23.528$\pm$0.013 & 0.0 & 9 & -    \\
0 & 2 & 2 & 1476.62 & 8.03     &   6.651480 &  17.129572 & 21.641$\pm$0.005  & 2 3.141$\pm$0.007 & 22.007$\pm$0.004 & 0.0 & 9 & -    \\
0 & 2 & 3 & 1079.62 & 28.92    &   6.652430 &  17.135000 & 23.953$\pm$0.030  & 2 4.890$\pm$0.023 & 24.240$\pm$0.023 & 0.0 & - & -    \\
0 & 2 & 4 & 114.58  & 21.22    &   6.655560 &  17.148020 & 23.801$\pm$0.025  & 2 5.039$\pm$0.026 & 24.112$\pm$0.021 & 0.0 & - & -    \\
0 & 2 & 5 & 46.78   & 19.46    &   6.655800 &  17.148932 & 23.012$\pm$0.012  & 2 3.924$\pm$0.012 & 23.282$\pm$0.011 & 0.0 & - & -    \\
0 & 2 & 6 & 1441.84 & 16.16    &   6.651480 &  17.130072 & 24.393$\pm$0.045  & 2 6.099$\pm$0.062 & 25.119$\pm$0.049 & 0.0 & - & -    \\
0 & 2 & 7 & 205.43  & 3.96     &   6.655520 &  17.146742 & 24.424$\pm$0.032  & 2 5.028$\pm$0.025 & 24.597$\pm$0.027 & 0.0 & - & -    \\
0 & 2 & 8 & 1321.63 & 9.76     &   6.651950 &  17.131672 &
22.189$\pm$0.011  & 2 4.743$\pm$0.021 & 23.298$\pm$0.011 & 0.0 & 4 &
edge \\
\hline\\[-6pt]
\multicolumn{13}{l}{%
Table 2 is published in its entirety in the electronic
edition of the {\it Astrophysical Journal}.}\\[3pt]
\multicolumn{13}{l}{%
$^a$ Sample footnote for table 2.}\\[3pt]
\multicolumn{13}{l}{%
$^b$ Another sample footnote for table 2.}
\end{tabular}
\end{widetable}
\end{table}

\begin{table}[p]
\rotatebox{90}{\vbox{\hsize=\textheight
\caption{Here is a caption for a table that is found in landscape
mode.}
\begin{tabular}{ccrrccccccccc}
\hline
\bf 
POS &\bf  chip &\multicolumn1c{\bf ID} &\multicolumn1c{\bf X}
&\multicolumn1c{\bf Y} &\bf
RA &\bf DEC &\bf IAU$\pm$ $\delta$ IAU &\bf
IAP1$\pm$ $\delta$ IAP1 &\bf IAP2 $\pm$ $\delta$
IAP2 &\bf star &\bf E &\bf Comment\\
\hline
0 & 2 & 1 & 1370.99 & 57.35\rlap{$^a$}    &   6.651120 &  17.131149 &
21.344$\pm$0.006\rlap{$^b$}  & 2 4.385$\pm$0.016 & 23.528$\pm$0.013 & 0.0 & 9 & -    \\
0 & 2 & 2 & 1476.62 & 8.03     &   6.651480 &  17.129572 & 21.641$\pm$0.005  & 2 3.141$\pm$0.007 & 22.007$\pm$0.004 & 0.0 & 9 & -    \\
0 & 2 & 3 & 1079.62 & 28.92    &   6.652430 &  17.135000 & 23.953$\pm$0.030  & 2 4.890$\pm$0.023 & 24.240$\pm$0.023 & 0.0 & - & -    \\
0 & 2 & 4 & 114.58  & 21.22    &   6.655560 &  17.148020 & 23.801$\pm$0.025  & 2 5.039$\pm$0.026 & 24.112$\pm$0.021 & 0.0 & - & -    \\
0 & 2 & 5 & 46.78   & 19.46    &   6.655800 &  17.148932 & 23.012$\pm$0.012  & 2 3.924$\pm$0.012 & 23.282$\pm$0.011 & 0.0 & - & -    \\
0 & 2 & 6 & 1441.84 & 16.16    &   6.651480 &  17.130072 & 24.393$\pm$0.045  & 2 6.099$\pm$0.062 & 25.119$\pm$0.049 & 0.0 & - & -    \\
0 & 2 & 7 & 205.43  & 3.96     &   6.655520 &  17.146742 & 24.424$\pm$0.032  & 2 5.028$\pm$0.025 & 24.597$\pm$0.027 & 0.0 & - & -    \\
0 & 2 & 8 & 1321.63 & 9.76     &   6.651950 &  17.131672 &
22.189$\pm$0.011  & 2 4.743$\pm$0.021 & 23.298$\pm$0.011 & 0.0 & 4 &
edge \\
\hline\\[-6pt]
\multicolumn{13}{l}{%
Table 2 is published in its entirety in the electronic
edition of the {\it Astrophysical Journal}.}\\[3pt]
\multicolumn{13}{l}{%
$^a$ Sample footnote for table 2.}\\[3pt]
\multicolumn{13}{l}{%
$^b$ Another sample footnote for table 2.}
\end{tabular}
}}
\end{table}
\clearpage
\begin{boxedtext}{Tools for comparison of networks} 
Going beyond the examination of shared topological features across
nervous systems, the generalized mathematical language of graph theory
also offers tools for the comparison of the organization of brain
networks to other classes of network studied by different scientific
disciplines. 

From $\mathcal{W}$, we can estimate the variability in the fluctuations of the functional connection between nodes $i$ and $j$ over time as:
\begin{equation}
s_{ij}=\sqrt{\frac{1}{T-L}\sum_{t=1}^{T-L+1}(W_{ij}(t) - m_{ij})}
\end{equation}
where $m_{ij}=\frac{1}{T-L+1}\sum_{t=1}^{T-L+1}W_{ij}(t)$ is the mean
dynamic functional connectivity over time. 

Many real-world systems operate as some sort of
interaction or communication network, including, for example, social
networks, gene regulatory networks, computer networks, and
transportation networks. Similar to brain networks, many of these
real-world networks display an efficient small-world organization, a
pronounced community structure with densely connected modules, as well
as the formation of hubs and rich clubs. Going beyond the
comparison of networks within the class of nervous systems, the field
of `comparative network analysis' examines commonalities and
differences across a range of network classes.
\end{boxedtext}

Example of table continuing over pages:


\begin{center}
\begin{longtable}{ccc@{}}
\caption{ApJ costs from 1991 to 2013
\label{tab:table}} \\[2pt]
\hline
\bf Year & \bf Subscription & \bf Publication \\
 & \bf cost &\bf charges\\
 & \bf(\$) & \bf (\$/page)\\
\hline
\endfirsthead

\multicolumn3c{Table \thetable, \it continued from previous page.}\\[6pt]
\multicolumn3c{ApJ costs from 1991 to 2013}\\[2pt]
\hline
\bf Year & \bf Subscription & \bf Publication \\
 & \bf cost &\bf charges\\
 & \bf(\$) & \bf (\$/page)\\
\hline
\endhead
\\\hline
\\[-8pt]
\multicolumn{3}{r}{\it Table continued on next page}\\ 
\endfoot

\hline
\endlastfoot

1991 & 600 & 100 \\
1992 & 650 & 105 \\
1993 & 550 & 103 \\
1994 & 450 & 110 \\
1995 & 410 & 112 \\
1996 & 400 & 114 \\
1997 & 525 & 115 \\
1998 & 590 & 116 \\
1999 & 575 & 115 \\
2000 & 450 & 103 \\
2001 & 490 &  90 \\
2002 & 500 &  88 \\
2003 & 450 &  90 \\
2004 & 460 &  88 \\
2005 & 440 &  79 \\
2006 & 350 &  77 \\
2007 & 325 &  70 \\
2008 & 320 &  65 \\
2009 & 190 &  68 \\
2010 & 280 &  70 \\
2011 & 275 &  68 \\
2012 & 150 &  56 \\
2013 & 140 &  55 \\
\end{longtable}
\end{center}

\acknowledgments
The Funder and award ID information you input at submission will be introduced by the publisher under a Funding Information head during production. 
Please use this space for any additional acknowledgements and verbiage required by your funders.

%% ie.,

% The authors thank Laurence Aitchison for fruitful discussions.  RKN
% and PD received funding from the Gatsby Charitable Foundation. Y-AB,
% RBS, KC and PS received funding from Canadian Institutes of Health
% Research grant $MOP74577$,
% Fond de recherche Qu\'{e}bec - Sant\'{e} (Group grant to the Groupe 
% de recherche en neurobiologie comportementale, Shimon Amir, P.I.), and
% Concordia University Research Chair (Tier I). 

\section{Supporting Information}
This is an optional section. Please use this space to provide information about any supporting information referred to in your manuscript.

\section{Competing Interests}
This is an optional section. If you declared a conflict of interest when you submitted your manuscript, please  use this space to provide details about this conflict.

\section{Making Your Bibliography for a Network Neuroscience Article}
{\it Network Neuroscience} uses the APA author-date  bibliography style,
apacite.bst. For more
information on apacite, for examples in how to make your .bib file and more, see:\\
\href{http://mirror.jmu.edu/pub/CTAN/biblio/bibtex/contrib/apacite/apacite.pdf}
{http://mirror.jmu.edu/pub/CTAN/biblio/bibtex/contrib/apacite/apacite.pdf}

\noindent
(In spite of the mention of apacite cite commands, please use only
Natbib commands for in text citations, as shown above.)

\subsection{BibTeX}
You will need to use BibTeX to form your bibliography; typing in the
references would be
a huge and unpleasant task. Look at the NETNSample.bbl file and you'll see why
typing in the bibitems would be difficult. 

For a good basic introduction to using BibTeX, see\\
\href{https://www.economics.utoronto.ca/osborne/latex/BIBTEX.HTM}
{https://www.economics.utoronto.ca/osborne/latex/BIBTEX.HTM}

When you use BibTeX, the form of the bibliography will be correct. You
don't need to supply a bibliography style, since that is built into
the stjour.cls file when the NETN option is used
(\verb+\documentclass[NETN]{stjour}+).



\newpage
\subsection{Sample citations}

Here are some samples using \verb+\citep{}+:\\
\citep{bullmore2009complex,
gomez2009analysis,
rubinov2011weight,
power2014methods,
sporns2015modular,
scheeringa2012eeg,
fortunato2007resolution,
reichardt2006statistical,
smith2009correspondence,
sporns2011networks,
liegeois2014cerebral}

And more using \verb+\citet{}+\\
\citet{allen2012tracking,
calhoun2014chronnectome,
liu2013time,
fisher1915frequency,
gonzalez2014spatial,
damaraju2014dynamic,
hutchison2013resting,
de2012cortical,
tagliazucchi2012dynamic,
nooner2012nki}

\newpage
%%%%%%%%%%%%%%%%%%%%%%%
%% The bibliography

\bibliography{NETNbibsamp}

\section{Technical Terms}

\textbf{Technical Term} a key term that is mentioned in an NETN article and whose usage and definition may not be familiar across the broad readership of the journal. 

\textbf{Technical Term} a key term that is mentioned in an NETN article and whose usage and definition may not be familiar across the broad readership of the journal. 

\textbf{Technical Term} a key term that is mentioned in an NETN article and whose usage and definition may not be familiar across the broad readership of the journal. 

\textbf{Technical Term} a key term that is mentioned in an NETN article and whose usage and definition may not be familiar across the broad readership of the journal. 


%% No appendices allowed in Network Neuroscience style
%\appendix

\end{document}

