%% NETN Template file

%% Documentclass:
\documentclass[NETN,manuscript]{stjour-new}

%% Or,
%% For production only, need .otf fonts, and
%% need to use lualatex:
%% \documentclass[finalfonts,NETN]{stjour}

%%%%%%%%%%% Article Set-Up %%%%%%%%%%%%%%%%%%%%%%%%%%%%%%%%%%%%
%% Article Type:

%% Choose one of these options, (Default is Research).
%% Research
%% Methods 
%% Data
%% Review
%% Perspectives

\articletype{Research}

%%%%%%%%%%%%%%%%%%%%%%%%%%%%%%%%%%%%%%%%%%%%%%%%%%%%%%%%%%%%%%%
%% author definitions should be placed here:

%% example definition
\def\taupav{\tau_{\mathrm{Pav}}}
\begin{document}

\title{<Title of Article>}
\subtitle{<Subtitle Here>} %% Optional subtitle

%% If shortened title for running head is needed so that the article title can fit
%%   in the running head, use [] argument, ie,
%%
%%   \title[Shortened Title for Running Head]{Title of Article}
%%   \subtitle{Subtitle Here}

%%% Author/Affil
%% Since we use \affil{} in the argument of the \author command, we
%% need to supply another version of the author names, without \affil{}
%% to be used for running heads:

\author[Author Names]% shortened version for running head, optional
{Author Names with affiliations\affil{1},
Another Name\affil{2}, Still another Name\affil{2},\\
\and Final Name\affil{1}}

\affiliation{1}{Department, Institution, City, Country}

%ie.
%\affiliation{1}{Gatsby Computational Neuroscience Unit, University
%College London, London, United Kingdom} 

\affiliation{2}{Another Department, Institution, City, Country}

%ie
%\affiliation{2}{Center for Studies in
%Behavioral Neurobiology, Concordia University, Montreal, Quebec,
%Canada}

\correspondingauthor{Author Name}{Corresponding author email address}

% ie,
%\correspondingauthor{Ritwik K. Niyogi}{ritwik.niyogi@gatsby.ucl.ac.uk}

\keywords{[a series of capitalized words, separated with commas]}

%ie
%\keywords{Work, Leisure, Normative, Microscopic,  Reinforcement Learning, Economics}

\begin{abstract}
Abstract text here.
\end{abstract}

\begin{authorsummary}
Author Summary here.
\end{authorsummary}

[Body of article. 
See NETNSample.tex for examples of all of the elements available for use in an NETN manuscript, including text boxes, lists, citations and references, and Technical Terms.]

%%%%%%%%%%%%%%%%%%%%%%%%%%%%%%%%%%%%%%%%%%%%%%%%%%%%%%%%%%%%%%%%%
%%% End of Article

% \acknowledgments
% \section{Supporting Information} (optional)
% \section{Competing Interests} (optional)
% \bibliography{<name of .bib file>}

\acknowledgments
The Funder and award ID information you input at submission will be introduced by the publisher under a Funding Information head during production. 
Please use this space for any additional acknowledgements and verbiage required by your funders.

\section{Supporting Information}
This is an optional section. Please use this space to provide information about any supporting information referred to in your manuscript.

\section{Competing Interests}
This is an optional section. If you declared a conflict of interest when you submitted your manuscript, please  use this space to provide details about this conflict.

\section{Technical Terms}

All NETN article types require Technical Terms.

Identify approximately 10 key terms that are mentioned in your article and whose usage and definition may not be familiar across the broad readership of the journal. 
Provide brief (20-word or less) definitions for each term, avoiding in these definitions the use of jargon, or highly technical or specialized language. 
When the article is typeset, the Technical Terms will appear in the margins at or near their first mention in the text.

In your manuscript, bold the first occurrence of each \textbf{Technical Term} and then provide a list of the terms and their definitions at the end of the manuscript after the references. 

%%%%%%%%%%%%%%%%%%%%%%%
%% The bibliography

%% The bibliography is made using only the entries that you cite using 
%% \cite{}, or one of the Natbib citation entries, like \citep{}, \citet{} etc.

\bibliography{<your .bib file>}

%ie.,
%\bibliography{bibsamp}

\section{Technical Terms}

\textbf{Technical Term} a key term that is mentioned in an NETN article and whose usage and definition may not be familiar across the broad readership of the journal. 

\textbf{Technical Term} a key term that is mentioned in an NETN article and whose usage and definition may not be familiar across the broad readership of the journal. 

\textbf{Technical Term} a key term that is mentioned in an NETN article and whose usage and definition may not be familiar across the broad readership of the journal. 

\textbf{Technical Term} a key term that is mentioned in an NETN article and whose usage and definition may not be familiar across the broad readership of the journal. 

\end{document}

The examples below may be helpful when you are looking
for a quick example. They are taken from NETNsample.tex
so you can see their results in NETNSample.pdf

\begin{boxedtext}{box title}
Text
\end{boxedtext}

or, put figure in boxed text:
\begin{boxedtext}{box title}
Text
\begin{figure}
illustration
\caption{caption here}
\end{figure}
\end{boxedtext}

or, put table in boxed text:
\begin{boxedtext}{box title}
Text
\begin{table}
\caption{caption here}
tabular...
\end{table}
\end{boxedtext}

or, put anything in whitebox within boxed text:
\begin{boxedtext}{Showing the use of whitebox} 
Going beyond the examination of shared topological features across..

\whitebox{
From $\mathcal{W}$, we can estimate the variability in the fluctuations of the functional connection between nodes $i$ and $j$ over time as:
\begin{equation}
s_{ij}=\sqrt{\frac{1}{T-L}\sum_{t=1}^{T-L+1}(W_{ij}(t) - m_{ij})}
\end{equation}
where $m_{ij}=\frac{1}{T-L+1}\sum_{t=1}^{T-L+1}W_{ij}(t)$ is the mean
dynamic functional connectivity over time. 
}
...differences across a range of network classes.
\end{boxedtext}

==================
More samples:

\section{Sample Section}
Text here. Text here. Text here. Text here.
Text here. Text here. Text here. Text here.
Text here. Text here. Text here. Text here.
Text here. Text here. Text here. Text here.

\subsection{Sample Subsection}
Text here. Text here. Text here. Text here.
Text here. Text here. Text here. Text here.
Text here. Text here. Text here. Text here.
Text here. Text here. Text here. Text here.

\subsubsection{Sample Subsubsection}
Text here. Text here. Text here. Text here.
Text here. Text here. Text here. Text here.
Text here. Text here. Text here. Text here.
Text here. Text here. Text here. Text here.

\section{Sample equations}
\begin{equation}
\label{eq:rhoCHT}
\rho^{\pi}= \frac{RI + \mathbb{E}_{\pi([L,\tau_L]|\textrm{post})}
\left[C_L(\taupav+\tau_L) \right]   +
\displaystyle{\int_{0}^{P}}{dw~ \mathbb{E}_{\pi_{w_L}}}
\Biggl[\/\sum_{n_{L|[\textrm{pre},w]}}C_L(\tau_L)
\Biggr]            }      {P +
\mathbb{E}_{\pi([L,\tau_L] |\textrm{post})}[\tau_{L}] +\taupav +
\displaystyle{ \int_{0}^{P}}{dw~ \mathbb{E}_{\pi_{w_L}}}   
\Biggl[\sum_{n_{L|[\textrm{pre},w]}}\tau_L\Biggr]  
}
\end{equation}
As long as
$RI - K_LP > 
\frac{1}{\beta}$
\begin{equation}
%\def\theequation{5.1}
\left.\begin{array}{lrcl}
&\rho^{\pi} &=&  \displaystyle\frac{\beta ( RI + K_L \taupav )-1} {\beta
(P+\taupav )}    \\[12pt]
\hbox{and}\hbox to .25in{\hfill}&\mathbb{E}[\tau_L | \text{post}] &=&\displaystyle \frac{P+\taupav}{\beta ( RI -
K_LP)-1}  
\label{eq:analytical_linear}
\end{array}\right\}\hbox to 1.25in{\hfill}
\end{equation} 

\subsection{Simple code sample}

\begin{code}
\begin{verbatim}
procedure bubbleSort( A : list of sortable items )
    n = length(A)
    repeat
       newn = 0
       for i = 1 to n-1 inclusive do
          if A[i-1] > A[i] then
             swap(A[i-1], A[i])
             newn = i
          end if
       end for
       n = newn
    until n = 0
end procedure
\end{verbatim}
\end{code}

\subsection{Algorithm environment}

%% \begin{algorithm} takes option [p][b][t][h],  or some combination, like \begin{figure}
%% See documentation for algorithmic.sty for more information on formatting algorithms.

\begin{algorithm}[h]
\caption{A sample in an algorithm environment.}
\begin{algorithmic}
\If {$i\geq maxval$}
    \State $i\gets 0$
\Else
    \If {$i+k\leq maxval$}
        \State $i\gets i+k$
    \EndIf
\EndIf
\end{algorithmic}
\end{algorithm}


\section{Technical Terms}
If you've been asked by the journal editor to provide technical terms and definitions, please bold the first occurrence of each term in the text and provide a list of the terms and their definitions at the end of the manuscript after the references. Please only supply technical terms if you've been instructed to do so by a journal editor.


\section{Itemized Lists}

\subsection{Roman list:}

\begin{enumerate}
\item[(i)] at high 
payoffs, subjects work almost continuously, engaging in little leisure
in between work bouts; 
\item[(ii)] at low payoffs, they 
engage in leisure all at once, in long bouts after working, rather
than distributing the same amount of leisure time into multiple short
leisure bouts; 
\item[(iii)] subjects work continuously for the entire price duration, as long as
the price is not very long (as shown by an analysis conducted by Y-AB, to be published separately); %(\textbf{Figure \ref{fig:task_data}D}).
\item[(iv)] the duration of leisure bouts is variable.
\end{enumerate}

\subsection{Numbered list:}

\begin{enumerate}
\item at high 
payoffs, subjects work almost continuously, engaging in little leisure
inbetween work bouts; 
\item at low payoffs, they 
engage in leisure all at once, in long bouts after working, rather
than distributing the same amount of leisure time into multiple short
leisure bouts; 
\item subjects work continuously for the entire price duration, as long as
the price is not very long (as shown by an analysis conducted by Y-AB, to be published separately); %(\textbf{Figure \ref{fig:task_data}D}).
\item the duration of leisure bouts is variable.
\end{enumerate}


\subsection{Bulleted list:}

\begin{itemize}
\item at high 
payoffs, subjects work almost continuously, engaging in little leisure
inbetween work bouts; 
\item at low payoffs, they 
engage in leisure all at once, in long bouts after working, rather
than distributing the same amount of leisure time into multiple short
leisure bouts; 
\item subjects work continuously for the entire price duration, as long as
the price is not very long (as shown by an analysis conducted by Y-AB, to be published separately); %(\textbf{Figure \ref{fig:task_data}D}).
\item the duration of leisure bouts is variable.
\end{itemize}

\section{Natbib citation mark up}

\subsection{Single citations}
\noindent
\begin{tabular}{ll}
\bf Type&\bf Results\\
\hline
\verb+\citet{jon90}+&Jones et al. (1990)\\
\verb+\citet[chap. 2]{jon90}+&Jones et al. (1990, chap. 2)\\
    \verb+\citep{jon90}+	    &   	(Jones et al., 1990)\\
    \verb+\citep[chap. 2]{jon90}+ 	&    	(Jones et al., 1990, chap. 2)\\
    \verb+\citep[see][]{jon90}+ 	 &    	(see Jones et al., 1990)\\
    \verb+\citep[see][chap. 2]{jon90}+ 	&    	(see Jones et al., 1990, chap. 2)\\
    \verb+\citet*{jon90}+ 	    &    	Jones, Baker, and Williams (1990)\\
    \verb+\citep*{jon90}+	    &    	(Jones, Baker, and Williams,
    1990) \\
\end{tabular}

For example, here are some sample citations:
citet:\citet{anderson}, citep:\citep{antibayes}, and
cite*: \citet*{anderson}.

\subsection{Multiple citations}
Multiple citations may be made by including more than one citation
key in the \verb+\cite+ command argument.

\noindent
\begin{tabular}{ll}
\bf Type&\bf Results\\
\hline
\verb+\citet{jon90,jam91}+&Jones et al. (1990); James et al. (1991)\\
\verb+\citep{jon90,jam91}+&(Jones et al., 1990; James et al. 1991)\\
\verb+\citep{jon90,jon91}+&(Jones et al., 1990, 1991)\\
\verb+\citep{jon90a,jon90b}+&(Jones et al., 1990a,b)\\
\end{tabular}

For example, multiple citations from the CompPsychSample bibliography:
citet:\citet{anderson,antibayes}, citep:\citep{anderson,antibayes}.
As you see, the citations are automatically hyperlinked to their
reference in the bibliography.

\newpage

\section{Sample figures}

\begin{figure}[h] 
\centerline{\includegraphics[width=\textwidth]{Fig1}}
\caption{(Colour online) \textbf{Task and key features of the
 data.} \\
 A) Cumulative handling time (CHT) task. Grey bars denote work
(depressing a lever), white gaps show leisure. The subject must
 accumulate work up to a total period of time called the
\emph{price} ($P$) in order to obtain a single reward (black dot) of subjective reward
intensity $RI$. The trial duration is $25\times \mathrm{price}$ (plus
$2$s each time the price is attained, during which the lever is retracted so it cannot
work; not shown).
}
\label{fig:task_data}
\end{figure}

\begin{figure}[ht] 
\widefigure{\fullpagewidth}{Fig1}
\caption{(Colour online) \textbf{Task and key features of the
 data.} \\
A) Cumulative handling time (CHT) task. Grey bars denote work
(depressing a lever), white gaps show leisure. The subject must
accumulate work up to a total period of time called the
\emph{price} ($P$) in order to obtain a single reward (black dot) of subjective reward
intensity $RI$. The trial duration is $25\times \mathrm{price}$ (plus
$2$s each time the price is attained, during which the lever is retracted so it cannot
work; not shown).
}
\label{newfig:task_data}
\end{figure}

\clearpage
\section{Sample tables}

\begin{table}[!ht]
\caption{Time of the Transition Between Phase 1 and Phase 2$^{a}$}
\label{tab:label}
\centering
\begin{tabular}{lc}
\hline
 Run  & Time (min)  \\
\hline
  $l1$  & 260   \\
  $l2$  & 300   \\
  $l3$  & 340   \\
  $h1$  & 270   \\
  $h2$  & 250   \\
  $h3$  & 380   \\
  $r1$  & 370   \\
  $r2$  & 390   \\
\hline
\multicolumn{2}{l}{$^{a}$Table note text here.}
\end{tabular}
\end{table}

\begin{table}[ht]
\widecaption{Sample table taken from [treu03]\label{tbl-1}}
\begin{widetable}
\advance\tabcolsep-1pt
\small
\begin{tabular}{ccrrccccccccc}
\hline
\bf 
POS &\bf  chip &\multicolumn1c{\bf ID} &\multicolumn1c{\bf X}
&\multicolumn1c{\bf Y} &\bf
RA &\bf DEC &\bf IAU$\pm$ $\delta$ IAU &\bf
IAP1$\pm$ $\delta$ IAP1 &\bf IAP2 $\pm$ $\delta$
IAP2 &\bf star &\bf E &\bf Comment\\
\hline
0 & 2 & 1 & 1370.99 & 57.35\rlap{$^a$}    &   6.651120 &  17.131149 &
21.344$\pm$0.006\rlap{$^b$}  & 2 4.385$\pm$0.016 & 23.528$\pm$0.013 & 0.0 & 9 & -    \\
0 & 2 & 2 & 1476.62 & 8.03     &   6.651480 &  17.129572 & 21.641$\pm$0.005  & 2 3.141$\pm$0.007 & 22.007$\pm$0.004 & 0.0 & 9 & -    \\
0 & 2 & 3 & 1079.62 & 28.92    &   6.652430 &  17.135000 & 23.953$\pm$0.030  & 2 4.890$\pm$0.023 & 24.240$\pm$0.023 & 0.0 & - & -    \\
0 & 2 & 4 & 114.58  & 21.22    &   6.655560 &  17.148020 & 23.801$\pm$0.025  & 2 5.039$\pm$0.026 & 24.112$\pm$0.021 & 0.0 & - & -    \\
0 & 2 & 5 & 46.78   & 19.46    &   6.655800 &  17.148932 & 23.012$\pm$0.012  & 2 3.924$\pm$0.012 & 23.282$\pm$0.011 & 0.0 & - & -    \\
0 & 2 & 6 & 1441.84 & 16.16    &   6.651480 &  17.130072 & 24.393$\pm$0.045  & 2 6.099$\pm$0.062 & 25.119$\pm$0.049 & 0.0 & - & -    \\
0 & 2 & 7 & 205.43  & 3.96     &   6.655520 &  17.146742 & 24.424$\pm$0.032  & 2 5.028$\pm$0.025 & 24.597$\pm$0.027 & 0.0 & - & -    \\
0 & 2 & 8 & 1321.63 & 9.76     &   6.651950 &  17.131672 &
22.189$\pm$0.011  & 2 4.743$\pm$0.021 & 23.298$\pm$0.011 & 0.0 & 4 &
edge \\
\hline
\multicolumn{13}{l}{%
Table 2 is published in its entirety in the electronic
edition of the {\it Astrophysical Journal}.}\\[3pt]
\multicolumn{13}{l}{%
$^a$ Sample footnote for table 2.}\\[3pt]
\multicolumn{13}{l}{%
$^b$ Another sample footnote for table 2.}
\end{tabular}
\end{widetable}
\end{table}

\begin{table}[p]
\rotatebox{90}{\vbox{\hsize=\textheight
\caption{Here is a caption for a table that is found in landscape
mode.}
\begin{tabular}{ccrrccccccccc}
\hline
\bf 
POS &\bf  chip &\multicolumn1c{\bf ID} &\multicolumn1c{\bf X}
&\multicolumn1c{\bf Y} &\bf
RA &\bf DEC &\bf IAU$\pm$ $\delta$ IAU &\bf
IAP1$\pm$ $\delta$ IAP1 &\bf IAP2 $\pm$ $\delta$
IAP2 &\bf star &\bf E &\bf Comment\\
\hline
0 & 2 & 1 & 1370.99 & 57.35\rlap{$^a$}    &   6.651120 &  17.131149 &
21.344$\pm$0.006\rlap{$^b$}  & 2 4.385$\pm$0.016 & 23.528$\pm$0.013 & 0.0 & 9 & -    \\
0 & 2 & 2 & 1476.62 & 8.03     &   6.651480 &  17.129572 & 21.641$\pm$0.005  & 2 3.141$\pm$0.007 & 22.007$\pm$0.004 & 0.0 & 9 & -    \\
0 & 2 & 3 & 1079.62 & 28.92    &   6.652430 &  17.135000 & 23.953$\pm$0.030  & 2 4.890$\pm$0.023 & 24.240$\pm$0.023 & 0.0 & - & -    \\
0 & 2 & 4 & 114.58  & 21.22    &   6.655560 &  17.148020 & 23.801$\pm$0.025  & 2 5.039$\pm$0.026 & 24.112$\pm$0.021 & 0.0 & - & -    \\
0 & 2 & 5 & 46.78   & 19.46    &   6.655800 &  17.148932 & 23.012$\pm$0.012  & 2 3.924$\pm$0.012 & 23.282$\pm$0.011 & 0.0 & - & -    \\
0 & 2 & 6 & 1441.84 & 16.16    &   6.651480 &  17.130072 & 24.393$\pm$0.045  & 2 6.099$\pm$0.062 & 25.119$\pm$0.049 & 0.0 & - & -    \\
0 & 2 & 7 & 205.43  & 3.96     &   6.655520 &  17.146742 & 24.424$\pm$0.032  & 2 5.028$\pm$0.025 & 24.597$\pm$0.027 & 0.0 & - & -    \\
0 & 2 & 8 & 1321.63 & 9.76     &   6.651950 &  17.131672 &
22.189$\pm$0.011  & 2 4.743$\pm$0.021 & 23.298$\pm$0.011 & 0.0 & 4 &
edge \\
\hline
\multicolumn{13}{l}{%
Table 2 is published in its entirety in the electronic
edition of the {\it Astrophysical Journal}.}\\[3pt]
\multicolumn{13}{l}{%
$^a$ Sample footnote for table 2.}\\[3pt]
\multicolumn{13}{l}{%
$^b$ Another sample footnote for table 2.}
\end{tabular}
}}
\end{table}
\clearpage


\vglue 3in
Example of table continuing over pages:


\begin{center}
\begin{longtable}{ccc@{}}
\caption{ApJ costs from 1991 to 2013
\label{tab:table}} \\[2pt]
\hline
\bf Year & \bf Subscription & \bf Publication \\
 & \bf cost &\bf charges\\
 & \bf(\$) & \bf (\$/page)\\
\hline
\endfirsthead

\multicolumn3c{Table \thetable, \it continued from previous page.}\\[6pt]
\multicolumn3c{ApJ costs from 1991 to 2013}\\[2pt]
\hline
\bf Year & \bf Subscription & \bf Publication \\
 & \bf cost &\bf charges\\
 & \bf(\$) & \bf (\$/page)\\
\hline
\endhead
\\\hline
\\[-8pt]
\multicolumn{3}{r}{\it Table continued on next page}\\ 
\endfoot

\hline
\endlastfoot

1991 & 600 & 100 \\
1992 & 650 & 105 \\
1993 & 550 & 103 \\
1994 & 450 & 110 \\
1995 & 410 & 112 \\
1996 & 400 & 114 \\
1997 & 525 & 115 \\
1998 & 590 & 116 \\
1999 & 575 & 115 \\
2000 & 450 & 103 \\
2001 & 490 &  90 \\
2002 & 500 &  88 \\
2003 & 450 &  90 \\
2004 & 460 &  88 \\
2005 & 440 &  79 \\
2006 & 350 &  77 \\
2007 & 325 &  70 \\
2008 & 320 &  65 \\
2009 & 190 &  68 \\
2010 & 280 &  70 \\
2011 & 275 &  68 \\
2012 & 150 &  56 \\
2013 & 140 &  55 \\
\end{longtable}
\end{center}

%%%%%%%%%%%%%%%%%%%%%%%
%% The bibliography

%% The bibliography is made using only
%% the entries that you cite using \cite{}, or one of the Natbib citation
%% entries, like \citep{}, \citet{} etc.

\bibliography{bibsamp}

NO APPENDICES allowed in the Network Neuroscience Style.

Please submit Supporting Information documents as PDFs in a format ready to publish.
