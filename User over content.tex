In Sides & Citrin's work, 

Even when information is corrected as quickly as possible -- in Thorson's work, the misinformation is corrected during the experiments -- recipients of untruths are likely to see their viewpoints shifted simply by being exposed to the untruths \citep{thorson2016belief}.


Therefore, 




In Bullock's first experiment, he gave a randomized group of participants a story about Medicaid in Wisconsin, with proposed changes that had "passed with 90 percent of one party voting for it and 90 percent of the other party voting against it." The parties were shuffled for each experiment, meaning that some had intuitive and counter-intuitive scenarios, such as the Republican party voting en masse against conservative changes. Regardless of which party the participant belonged to, there was a statistically significant (p < 0.05) change in overall response based off of party cues -- Democratic participants were more likely to reject liberal changes and Republican participants were more likely to reject conservative changes if they had read that the party they identified with also rejected those changes \citep{bullock2007experiments}. This reinforces that the words spoken are less important than the speaker. 



 Republicans and Democrats are likely to make their first decision on whether or not a statement is true based on if the statement is coming from a preferred politician. 