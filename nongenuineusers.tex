Network Theory.

\section{Echo Chambers}
 *** Welch story ***
 
 This is entirely in line with current network theories of social media: because social media seeks to connect like minded individuals, it has a tendency to generate \textit{echo chambers}, or closed networks with high repetition of content and low diversity of thought \citep{adibi2005proceedings, bastian2009international, pariser2011filter,bozdag2015breaking}. While not every closed homogeneous network is inherently bad -- they can operate as self-help groups a \citep{kast2012under} or can encourage positive behavior like reducing prejudice \citep{paluck2011peer} -- there is a clear and well documented history that people are influenced by others in their network \citep{bollinger2012peer, bond201261,gerber2008social,gerber2009descriptive,meer2011brother,paluck2012salience,del2016spreading} and can be destructive in the context of misinformation.
 
 These \textit{echo chambers} create a tribal mentality, which we see in Bullock's experiments, where he found that participants would change their opinions on a particular topic if they were told that the party they identify with held an opposing view, even if it was counter-intuitive, such as a Republican being told that the Republican party was against a conservative initiative \citep{bullock2007experiments}. Other research provides similar results: Republicans and Democrats are likely to accept a statement as being true and not feel the need to research personally if it comes from a preferred politician \citep{housholder2014facebook}; even if a preferred politician's statements are disproved, there is no shift in voting intentions, party identification, or overall perceived credibility of that politician \citep{swire2017processing}. 
 Further complicating the situation, closed networks are likely to be more resistant to fact-checking or corrective information coming from outside of their network \citep{garrett2013undermining}. This explains why even the release of President Obama's long form birth certificate only briefly quieted the conspiracy that he was not born in the United States \citep{nyhan2012new} -- so long as the corrective information came from outside of the echo-chamber, it was fundamentally ineffective.
 
 \subsection{Non-genuine Users}
 Non-genuine users (NGUs) are users that are not who they claim to be. This is a wide term meant to encompass bots