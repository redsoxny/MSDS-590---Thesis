One out-sized issue is that 


In an analysis of 2018 data from Twitter, there was a significant difference between the number of times a tweet with an article was retweeted vs. clicked, with a huge disparity in the first 24 hours and the ratio gradually approaching 1 after 96 hours \citep{holmstrom2019we}. This phenomenon of people retweeting more than clicking is exacerbated by using bit.ly links -- a tool to shorten a longer url string -- which have become popular on mediums like Twitter where character counts are limited \citep{mostrom2020longitudinal}.





that would utilize an excessive amount of compute power and, therefore, would be highly cost ineffective for a social media company. On top of this, validating every tweet containing a URL is unnecessary as only 59 percent of all tweeted URLs are ever clicked \citep{gabielkov2016social}. This number is misleading, however, in an attempt to gauge overall interactions, as if a user sees the same link from multiple locations, they are unlikely to click multiple times. However, Gabielkov et al. saw that when they removed duplicate receptions to the same user, they could utilize a CPF ratio (click-per-follower) to predict an account's number of followers based off of the average clicks **** symbol ****. This ratio was able to accurately predict 75 percent of the receptions with less than 20 percent error. 

Gabielkov et al. also discovered that primary URLs (a URL that is generated by a news source and is disseminated from an official news account) account for 2 percent of the total URLs on a topic and all of them are clicked at least once, generating 39 percent of the total clicks on a topic. Secondary URLs (a URL that is \textit{not} from a primary news source from an official news account), however, generate the majority of total clicks on a topic (61 percent), with about 7 percent of these secondary URLs capturing 50 percent of the click traffic. In other words, 9 percent of the total URLs on a topic generate 90 percent of the total click traffic.


Social media and web related networks have a surprisingly small diameter -- the average distance \textit{d} between two particular pages on the internet is slightly less than 19 clicks \citep{albert1999diameter} -- which reinforces that 




While this "tit-for-tat" or "mutually assured destruction" strategy is a highly successful strategy in game theory \citep{segal2007tit}, a fairness equilibrium can be generated \citep{rabin1993incorporating}, such that Think Progress's strategy for fact-checking, $\sigma_t$, and the Weekly Standard's strategy for fact-checking, $\sigma_w$ would both skew towards being lenient in the belief that the other side would therefore be lenient towards them. Indeed, even though groups struggle with recognizing which stores are false when those stories are about an opposing political party, they are much better at capturing mis-characterizations about their own party \citep{pennycook2019lazy}. A scenario wherein users with $\rho = 0$ fact-check those with $\rho = 1$ and vice-versa could potentially be an effective solution, assuming that 

This inability to accurately assess an article fundamentally throws off the fact-checking "strategies" that Segal and Rabin discuss, and this would be an interesting topic for further analysis.




%%% Equation with weightings

 \begin{equation}
    \label{ech chamber}
        \langle \rho_n \rangle = \frac{1}{\sum_{i=1}^{|N_n|}}\sum_{i=1}^{|N_n|}{\rho_i}{k_i}, (\rho \in \Rho: 0 \leq \rho \leq 1) 
 \end{equation}
 \begin{equation}
    \label{leaningproportionaltonetwork}
        \rho_k \propto \langle \rho_n \rangle
 \end{equation}
 
 
 
 
Charlie Kirk's tweet needs to be examined before Bernie Sanders's tweet, from a mathematical perspective. In the examples provided, Charlie Kirk's tweet had over 34,100 retweets vs. Sanders's 27,200. While $r_k > r_s$, the more important factor is that Kirk has 1,750,000 followers, whereas Sanders has 14,450,000. Given the ratio:
\begin{equation}
\label{retweetsperfollower}
    \lambda = \frac{r}{|Y|}
\end{equation}
Then the $\lambda$ value for each tweet becomes:
\begin{equation}
    \begin{split}
        \lambda = \frac{34,100}{1,750,000}=0.019 \\
        \lambda = \frac{27,200}{14,450,000}=0.0019
    \end{split}
\end{equation}

$\lambda_k \approx 10\lambda_s$


Pastor-Satorras and Vespignani recognized that for a consistently spreading disease with a rate of transmission $\omega$ and a number of nodes $|U|$,:
  \begin{equation}
    g_c = \exp{\frac{-2}{|U|\omega}}
\end{equation}

This comes from 

However, there is no inherent correlation between number of followers and number of retweets. 